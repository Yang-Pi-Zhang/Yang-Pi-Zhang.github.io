\documentclass{cv} % Use the custom resume.cls style

\usepackage[left=0.75in,top=0.6in,right=0.75in,bottom=0.6in]{geometry}
\usepackage{amssymb}
% Document margins
\newcommand{\tab}[1]{\hspace{.2667\textwidth}\rlap{#1}}
\newcommand{\itab}[1]{\hspace{0em}\rlap{#1}}





\name{Yikai Teng} % Your name
\address{212 East Green St, Apt 407, Champaign, Illinois.} % Your address
%\address{123 Pleasant Lane \\ City, State 12345} % Your secondary addess (optional)
\address{(217)3051137 \\ yikait2@illinois.edu} 
\address{Personal homepage: yikaiteng.net}% Your phone number and email



\begin{document}
%----------------------------------------------------------------------------------------
%	EDUCATION SECTION
%----------------------------------------------------------------------------------------

\begin{rSection}{Education}

{\bf University of Illinois at Urbana-Champaign} \hfill {\em August 2018 - Present} 
\\ Major in Mathematics, Minor in Computer Science \hfill {Expected: May 2021}
\\ Dean's List: Fall 2018 - Spring 2020 \hfill {GPA: 3.94} 
% \\{\bf Maharashtra Institute of Techology, Pune} \hfill {\em July 2013 - June 2017} 
% \\ Bachelor of Engineering, Civil.\hfill { Overall Percentage: 68.14 }
% %Minor in Linguistics \smallskip \\
% %Member of Eta Kappa Nu \\
% %Member of Upsilon Pi Epsilon \\
\end{rSection}


\begin{rSection}{Academic Interests}
    I am interested in Differential Geometry and Geometric Topology, in particular, low-dimensional topology, 3-manifolds, and knot theory.
\end{rSection}




\begin{rSection}{Teaching Experiences}

{\bf Course Aide: Numerical Methods} \hfill {\em August 2019 - Present}
\\ \textbullet{} Teach theory behind numerical methods and help around 300 students in this Python-based course.
\\ \textbullet{} Assist the professor with in-class activities and help develop and revise homework assignments.
\\ \textbullet{} Host weekly office hours, aid around 20 students with their academic problems.
\\ \textbullet{} Help create and revise online notes for numerical methods and build class website.

\end{rSection}





%--------------------------------------------------------------------------------
%    Projects And Seminars
%-----------------------------------------------------------------------------------------------
\begin{rSection}{Researches and Projects}
{\bf Polymath REU} \hfill {\em June 2020 - Present}
\\ {\em Group Research Project, Under Postdoc Cody Stockdale, Clemson University.}
\\ \textbullet{} Work on finding a bound for weak-type $(1,1)$ property for Riesz Transforms.
\\ \textbullet{} Compose a geometric proof for a dimension dependent bound.

{\bf Independent Research} \hfill {\em January 2020 - Present}
\\ {\em Independent Research Project, Under Professor Bruce Reznick, U of I.}
\\ \textbullet{} Work on the generalization of Arnold's Cat Map on various dimensions and spaces.
\\ \textbullet{} Focus on the group of measure preserving self maps on arbitrary dimensional tori $\mathbb{T}^n$, and its relation to the mapping class group, $\mathcal{MCG}(\mathbb{T}^n)$.
\\ \textbullet{} Create Mathematica visualizations of generalized Cat Map.

{\bf Illinois Geometry Lab} \hfill {\em January 2019 - September 2019}
\\ {\em Visualization Project, Under Professor AJ Hildebrand, U of I.}
\\ \textbullet{} Study the coupon collector problem and coupon collector randomness test and visualize such problems by Mathematica.
\\ \textbullet{} Simulation accepted by Wolfram demonstration.


\end{rSection}






%----------------------------------------------------------------------------------------
%	TECHNICAL STRENGTHS SECTION
%----------------------------------------------------------------------------------------

\begin{rSection}{ExtraCurricula Courses}

{\bf Complex Algebraic Curves} \hfill {\em January 2020 - Present}
\\ {\em Reading Project, Under Professor Steven Bradlow, U of I.}
\\ \textbullet{} Study the foundation and properties of complex algebraic curves, both algebraically and topologically, including B\'ezout's Theorem, the degree-genus formula, etc.
\\ \textbullet{} Study complex algebraic curves as Riemann surfaces and related theorems like Abel's Theorem and the Riemann-Roch Theorem.

{\bf Complex Analysis in a Geometric Approach} \hfill {\em August 2019 - December 2019}
\\ {\em Reading Project, Under Professor Richard Laugesen, U of I.}
\\ \textbullet{} Apply classical Complex analysis in geometry to study particular metrics like Poincare, Carathe\'{o}dory, and Kobayashi metric.
\\ \textbullet{} Compare the geometry in complex analysis with classical differential geometry to study the cross sections of the two fields.
\\ \textbullet{} Study harmonic mappings in the complex domain and its application to minimal surface theory.

{\bf Modern Theory of Dynamical Systems} \hfill {\em August 2019 - December 2019}
\\ {\em Reading Project, Under Professor Eduard-Wilhelm Kirr, U of I}
\\ \textbullet{} Study advanced modern theory of dynamical systems, particularly the behavior around a hyperbolic fixed point, like the Hadamard-Perron Theorem and the Hartman-Grobman Theorem.
\\ \textbullet{} Conduct the proof of the existence of the Lake of Wada as a group of four.

\end{rSection}

%----------------------------------------------------------------------------------------
%	WORK EXPERIENCE SECTION
%----------------------------------------------------------------------------------------

\begin{rSection}{Additional Activities}

{\bf Mathematical Contest for Modelling} \hfill {\em February 2020}
\\ \textbullet{} Construct a math model to predict where Scottish herring will migrate for the next few decades and provide suggestions for fishing companies in Scotland.
% \\ \textbullet{} Win the Honorable Mention's Award

{\bf Mechmania} \hfill {\em Septempber 2019}
\\ \textbullet{} Develop a strategy of a board game to compete with other contestants.

{\bf HackIllinois} \hfill {\em February 2019}
\\ \textbullet{} Compose tests for LinearMappings package for Julia and add support for Quaternion Numbers.
\\ \textbullet{} Compose various tests and make multiple contributions to the DoubleFloat package for Julia.


\end{rSection}

\begin{rSection} {References}

\end{rSection}


% %	EXAMPLE SECTION
% %----------------------------------------------------------------------------------------

% \begin{rSection}{Academic Achievements} 
%  Runners up in B.G.Shirke Vidyarthi Competition for Innovative Project organized by Pune Construction Engineering Research Foundation in January 2018
% \item Won First Prize in Model Making Competition Organized by Symbiosis Institute of Technology, Pune.
% \end{rSection}

% \newpage

% %----------------------------------------------------------------------------------------
% % Extra Curricular
% %----------------------------------------------------------------------------------------
% \begin{rSection}{Extra-Cirrucular} \itemsep -3pt
% \item Co-Organized “ Nirmitee 2017” - a National Symposium of Civil Department of MIT, Pune
% \item Attended a workshop on Autodesk Revit at IIT Bombay in 2014.
% \item Winner of Inter Departmental Football Competition 2015.
% \item Member of the  Rotaract Club Of Pune Pride from 2014 to 2017.
% \item Worked for a start-up company Named OUST as a Regional Marketing Manager
% %\item Trained and disciplined in National Cadet Corps (NCC), IIT Kanpur for a year.
%  %\item  Participated in Vijyoshi Camp 2012 organized at Indian Institute of Science, Bangalore.
%  %\item Won 2nd position in Kho-Kho in Intramurals conducted by Physical Education Section, IIT Kanpur.
%  %\item Pursued French as second language during secondary school from Grade 6 to Grade 10. Also participated in French Song Competition and French G.K. Quiz in Class 10th. %

% \end{rSection}

% \begin{rSection}{Personal Traits}
% \item Highly motivated and eager to learn new things.
% \item Strong motivational and leadership skills.
% \item Ability to work as an individual as well as in group.
% \end{rSection}
\end{document}

